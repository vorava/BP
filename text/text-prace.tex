\chapter{Úvod}
Neuronové sítě (anglicky neural networks) mají v dnešním světě mnoho využití. Jelikož se jedná o jednu z aplikací umělé inteligence (anglicky artificial intelligence), lze neuronové sítě použít například k rozpoznávání řeči, zpracování přirozeného jazyka či k detekci objektů.
Tyto akce jsou pro běžného člověka poměrně snadné, avšak pro počítače znamenají relativně náročnou činnost.

Aby byly počítače schopné tyto akce vykonávat v rozumném čase (případně v reálném čase), je potřeba aby neuronové sítě byly dostatečně rychlé. Tato práce se zabývá akcelerací neuronových sítí v oblasti detekce obličeje. Zrychlení neuronové sítě lze dosáhnout buď optimalizací kódu, lepším trénováním neuronové sítě nebo také využitím speciálních hardwarových zařízení. Jedním z těchto specializovaných zařízení je Intel Neural Compute Stick 2, na něž se v této práci zaměřím. 

Následující kapitola se obecně věnuje problematice detekce obličeje v reálných podmínkách s využitím neuronových sítí. Je zde detailně popsáno \todo{CO JE DETAILNě Popsáno?}.

V kapitole \ref{kapitola:navrh_reseni} je nastíněn návrh akcelerace neuronové sítě pro detekci obličeje. \todo{\blindtext}

Kapitola \ref{kapitola:implementace} se věnuje implementaci programu ke zrychlení detekce, také obsahuje informace o využitých prostředcích.

Předposlední kapitola (kapitola č.\ref{kapitola:porovnani_vykonnosti}) poskytuje přehled experimentů a testů provedených s implementovaným řešením a s řešeními již existujícími. Hlavním tématem v této části je porovnání výkonnosti a zobrazení dosažených výsledků.

\nocite{*}
%%%%%%%%%%%%%%%%%%%%%%%%%%%%%%%%%%%%%%%%%%%%%%%%%%%%%%%%%
%       KAPITOLA 2
%%%%%%%%%%%%%%%%%%%%%%%%%%%%%%%%%%%%%%%%%%%%%%%%%%%%%%%%%

\chapter{Problematika detekce obličeje pomocí neuronových sítí}
\label{kapitola:problematika_detekce}
\todo{\blindtext}
todo

\section{Neuronové sítě}
\todo{\blindtext}

\section{Detekce obličeje}
\todo{\blindtext}

\section{Existující způsoby akcelerace detekce}
\todo{\blindtext}


\section*{Intel Neural Compute Stick 2}
\todo{\blindtext}

\section{Data pro detekci}
\todo{\blindtext}

\begin{figure*}[h]\centering
  \centering
  \includegraphics[width=\linewidth,height=1.7in]{obrazky-figures/placeholder.pdf}\\[1pt]
  \label{TODO}
  \caption{TODO}
\end{figure*}


%%%%%%%%%%%%%%%%%%%%%%%%%%%%%%%%%%%%%%%%%%%%%%%%%%%%%%%%%
%       KAPITOLA 3
%%%%%%%%%%%%%%%%%%%%%%%%%%%%%%%%%%%%%%%%%%%%%%%%%%%%%%%%%

\chapter{Návrh řešení zrychlení detekce obličeje}
\label{kapitola:navrh_reseni}
\todo{\blindtext}

\section{Zhodnocení současného stavu}
\todo{\blindtext}

\section{Návrhy vylepšení}
\todo{\blindtext}


%%%%%%%%%%%%%%%%%%%%%%%%%%%%%%%%%%%%%%%%%%%%%%%%%%%%%%%%%
%       KAPITOLA 4
%%%%%%%%%%%%%%%%%%%%%%%%%%%%%%%%%%%%%%%%%%%%%%%%%%%%%%%%%

\chapter{Implementace algoritmu pro akceleraci detekce obličeje}
\label{kapitola:implementace}
\todo{\blindtext}

\section{Použité nástroje}
\todo{\blindtext}

\section{Trénování neuronové sítě}
\todo{\blindtext}

\section{Detekce obličejů natrénovanou sítí}
\todo{\blindtext}


%%%%%%%%%%%%%%%%%%%%%%%%%%%%%%%%%%%%%%%%%%%%%%%%%%%%%%%%%
%       KAPITOLA 5
%%%%%%%%%%%%%%%%%%%%%%%%%%%%%%%%%%%%%%%%%%%%%%%%%%%%%%%%%

\chapter{Porovnání výkonnosti řešení s existujícími detektory}
\label{kapitola:porovnani_vykonnosti}
\todo{\blindtext}

\section{Postup testování}
\todo{\blindtext}
\begin{figure*}[h]\centering
  \centering
  \includegraphics[width=\linewidth,height=1.7in]{obrazky-figures/placeholder.pdf}\\[1pt]
  \label{TODO}
  \caption{TODO}
\end{figure*}

\section{Porovnání výsledků}
\todo{\blindtext}

\section{Shrnutí}
\todo{\blindtext}

%%%%%%%%%%%%%%%%%%%%%%%%%%%%%%%%%%%%%%%%%%%%%%%%%%%%%%%%%
%       KAPITOLA 6
%%%%%%%%%%%%%%%%%%%%%%%%%%%%%%%%%%%%%%%%%%%%%%%%%%%%%%%%%
\chapter{Závěr}
\label{kapitola:zaver}
\todo{\Blindtext}